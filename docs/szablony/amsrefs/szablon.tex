\documentclass[brudnopis]{xmgr}
\usepackage{amsrefs}
%
% kompilacja:
%
%   xelatex szablon
%   xelatex szablon

% tak można zmienić domyślną rodzinę fontów Latin Modern 
%   na Minion + Myriad:
%\defaultfontfeatures{Scale=MatchLowercase}
%\setmainfont[Mapping=tex-text]{Minion Pro:+onum}
%\setsansfont[Mapping=tex-text]{Myriad Pro}
%\setmonofont[Scale=0.75]{Monaco}

\wersja   {wersja wstępna [\ymdtoday]}

\author   {Imię Nazwisko}
\nralbumu {xxx\,xxx}
\email    {login@inf.ug.edu.pl}

\title    {Tytuł pracy}
\date     {????}
\miejsce  {Gdańsk}

\opiekun  {prof. dr hab. D. Htunk}

\begin{document}

\begin{abstract}
  Streszczenie…
\end{abstract}
\keywords{Ruby, 
 programowanie, 
 dokumenty strukturalne}

% tytuł i spis treści
\maketitle

% wstęp
\introduction

Jak pisać pracę magisterską?

\url{http://sinatra.inf.ug.edu.pl/seminarium/info/jak-pisac}.

Dokumentację pakietu \emph{amsref} można znaleźć tutaj:
\url{http://www.ctan.org/tex-archive/macros/latex/contrib/amsrefs/}.

\chapter{Tytuł rozdziału}

SGML~\cite{Goldfarb2002} jest to \emph{metajęzyk} 
służący do opisywania struktury i~zawartości dokumentów.
    
\section{Tytuł rozdziału}

Typowy proces produkcji dokumentów w~standardzie SGML
podzielony jest na kilka części. Najważniejsze elementy tego
procesu są opisane w~\cite{Eisenberg2002}.

\subsection{Tytuł podrozdziału}

\summary

Streszczenie…

% załączniki (opcjonalnie):
\appendix
\chapter{Tytuł załącznika}

Załącznik…

\chapter{Tytuł załącznika}

Załącznik…

% literatura
\begin{bibdiv}
\begin{biblist}

  \bibselect{literatura}

\end{biblist}
\end{bibdiv}

% spis tabel
%\listoftables

% spis rysunków
%\listoffigures

\oswiadczenie

\end{document}
