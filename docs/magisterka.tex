\documentclass[brudnopis]{xmgr}
% Jeśli nowe rozdziały mają się zaczynać na stronach
% nieparzystych:
%\documentclass[openright]{xmgr}

\usepackage{epigraph}
\usepackage{url}

\defaultfontfeatures{Scale=MatchLowercase}
\setmainfont[Numbers=OldStyle,Ligatures=TeX]{Minion Pro}
\setsansfont[Numbers=OldStyle,Ligatures=TeX]{Myriad Pro}
% for fontspec version < 2.0
%\setmainfont[Numbers=OldStyle,Mapping=tex-text]{Minion Pro}
\setsansfont[Numbers=OldStyle,Mapping=tex-text]{Myriad Pro}
%\setmonofont[Scale=0.75]{Monaco}

% Opcjonalnie identyfikator dokumentu
% drukowany tylko z włączoną opcją 'brudnopis':
\wersja   {wersja wstępna [\ymdtoday]}

\author   {Piotr Lewandowski}
\nralbumu {215575}
\email    {poczta@piotrl.net}

\title    {Analiza wpływu nawyków muzycznych na aktywności wykonywane przy komputerze}
\date     {2017}
\miejsce  {Gdańsk}

\opiekun  {dr W. Bzyl}

% dodatkowe polecenia
%\renewcommand{\filename}[1]{\texttt{#1}}
%\definecolor{stress}{cmyk}{0,1,0.13,0} % RubineRed
%\definecolor{topic}{cmyk}{0.98,0.13,0,0.43} % MidnightBlue

\begin{document}

% streszczenie
\begin{abstract}
    W ramach pracy magisterskiej napisano aplikację internetową,
    wdrożoną w chmurze Digital Ocean pod adresem http://tbd.digitalocean.com/
    z przygotowanymi danymi testowymi pod kontem (user: test, login: test).

    Aplikacja umożliwia agregację danych użytkownika z dwóch serwisów,
    RescueTime — lista aktywności oraz
    Last.fm — lista odsłuchiwanych utworów.
    Pobrane dane są połączone na wspólnej osi czasu i wizualizowane pod różnymi względami za pomocą wykresów oraz tabel.

    Mechanizm agregacji napisany jest w języku Java i frameworku Spring,
    dane przechowywane są w bazie danych PostgreSQL,
    a warstwa wizualna została stworzona w JavaScript
    z użyciem biblioteki generującej wykresy — C3.js
    oraz frameworka Material Design.

    Kod znajduje się w prywatnym repozytorium GIT (pod adresem \url{https://github.com/piotrl/master-thesis}),
    pytania o dostęp lub o pracę można kierować na mail: poczta@piotrl.net.
\end{abstract}

% słowa kluczowe
\keywords{
    Aggregation,
    Data Integration,
    Data Visualisation,
    PostgreSQL,
    JavaScript,
    Java
}

% tytuł i spis treści
\maketitle

% wstęp
\introduction

\epigraph{Without deviation from the norm, progress is not possible}{\textit{Frank Zappa}}

[Work in progress]

Podczas pracy, często słucham muzyki by się odciąć od zewnętrznych dźwięków oraz odpowiednio nastroić.

Spotyka się na artykuły twierdzące, że odpowiednio dobrana muzyka wpływa na lepszą koncentrację.
Popularnym przykładem jest wystąpienie Willa Henshalla na konferencji TEDx,
którą w momencie pisania pracy obejrzano ponad 400 tysiący razy:

\begin{figure}
  \includegraphics[width=\linewidth]{fig/tedx-will-hensell.png}
  \caption{Tedx Bruksela 04.11.2013: }
  \label{fig:Tedx Bruksela 04.11.2013: }
\end{figure}

Jednym z pytań w corocznej sondzie serwisu StackOverflow jest pytanie o muzyczne preferencje podczas programowania.
Na to pytanie odpowiedziało XX tysięcy ludzi (TODO: Źródło).

\begin{figure}
  \includegraphics[width=\linewidth]{fig/stack_overflow_music.png}
  \caption{Pytanie o preferncje muzyczne na StackOverflow}
  \label{fig:Pytanie o preferncje muzyczne na StackOverflow}
\end{figure}


\chapter{Agregacja danych}

    \section{Wybór dostawców danych}

    Podstawą działania aplikacji są dane z komputera użytkownika,
    lista aktywności wykonywanych w danej jednostce czasu oraz odsłuchiwana muzyka.
    Aby zdobyć te dane, postanowiłem nie pisać programu śledzącego procesy (daemon),
    lecz skorzystać z gotowych rozwiazań.

        \subsection{RescueTime - zasada działania}

        RescueTime to program który działa w tle, nasłuchuje na aktualnie aktywne procesy
        i raz na minutę wysyła te dane do serwisu internetowego RescueTime.com.
        Podstawową cechą tego serwisu jest automatyczne kategoryzowanie aktywności oraz ustawienie produktywności w 5 punktowej skali:
        „Bardzo produktywny”, „Produktywny”, „Neutralny”, „Rozpraszający” i „Bardzo rozpraszający”.

        Dla przykładu, odtwarzacz wideo VLC jest przypisany do kategorii rozrywka ze statusem „Bardzo rozpraszający”,
        natomiast przeglądanie plików programem Windows Explorer jest w kategorii „Pomocnicze” ze statusem „Neutralny”.

        Dzięki temu, jesteśmy w stanie sprawdzić nasz codzienny współczynnik produktywności.

        Program jest w stanie rozpoznać nie tylko tytuły aktualnie otwartych okien,
        ale również w przypadku przeglądarki – adresy odwiedzanych stron internetowych,
        jest to bardzo ważne ze względu na dominującą ilość czasu spędzonego dziennie w samej przeglądarce.

        \subsection{last.fm - zasada działania}

        Last.fm to serwis społecznościowy dla fanów muzyki,
        który udostępnia wtyczki do popularnych odtwarzaczy muzycznych, np. Spotify.
        Każdy odsłuchany w co najmniej w połowie utwór jest logowany do bazy danych,
        w której użytkownicy na podstawie meta-tagów przypisują go do artysty, płyty oraz gatunku muzycznego.

        Oba z wymienionych serwisów dostarczają API z danymi jakie użytkownik zgromadził na swoim koncie.


    \section{Cechy mechanizmu agregacji}

    \subsection{Niezależność}

        \subsection{Ciągłość}

        \subsection{Powtarzalność}

    \section{Powszechne problemy przy komunikacji z API}

        \subsection{Ograniczenia dostawcy API}

        \subsection{Autoryzacja}

        \subsection{Braki dostępu}

\chapter{Analiza danych}

    \section{Opis zebranych danych}

    \section{Data fusion - Łączenie danych w czasie}

        \subsection{Korekta}

        \subsection{Integracja dwóch źródeł danych}

     \section{Budowa raportów na podstawie szeregów czasowych}

\chapter{Wizualizacja danych}

     \section{Jeden zbiór danych - wiele interpretacji}

     \section{Wybór odpowiednich metod wizualizacji danych}

     \section{Analiza danych z punktu widzenia użytkownika}

\chapter{MOCK DATA BELOW}

\chapter{Przegląd dostępnych narzędzi\label{PRZEGLAD.NARZEDZI}}

W~celu wykorzystania standardu SGML do przetwarzania dokumentów,
niezbędne jest zebranie odpowiedniego zestawu narzędzi. Narzędzi do
przetwarzania dokumentów SGML jest wiele. Są to zarówno całe
systemy zintegrowane, jak i~poszczególne programy, biblioteki czy
skrypty wspomagające.

\section{Narzędzia do przeglądania dokumentów SGML}

Do tej kategorii oprogramowania zaliczamy przeglądarki dokumentów
SGML oraz serwery sieciowe wspomagające standard SGML, przy
czym rozwiązań wspierających standard XML jest już w~chwili obecnej
dużo więcej i~są dużo powszechniejsze.

Jeżeli chodzi o~przeglądarki to zarówno Internet Explorer jak
i~Netscape umożliwiają bezpośrednie wyświetlenie dokumentów XML;
ponieważ jednak nie wspierają w~całości standardu XML, prowadzi to
ciągle do wielu problemów\footnote{Z~innych mniej popularnych
  rozwiązań można wymienić takie aplikacje, jak: HyBrick SGML
  Browser firmy Fujitsu Limited, Panorama Publisher firmy InterLeaf
  Inc, DynaText firmy Inso Corporation czy darmowy QWeb. W~przypadku
  serwerów zwykle dokonują one transformacji ,,w~locie'' żądanych
  dokumentów na format HTML (rzadziej bezpośrednio wyświetlają
  dokumenty XML).  Ta kategoria oprogramowania ma, z~punktu widzenia
  projektu, znaczenie drugorzędne.}.

\section{Parsery SGML}
Program \texttt{nsgmls} (z~pakietu \texttt{SP} Jamesa Clarka) jest
doskonałym parserem\index{parser} dokumentów SGML, dostępnym
publicznie.  Parser \texttt{nsgmls} jest dostępny w~postaci źródłowej
oraz w~postaci programów wykonywalnych przygotowanych na platformę
MS~Windows, Linux/Unix i~inne. Oprócz analizy poprawności dokumentu
parser\index{parser} ten umożliwia również konwersję danych do formatu
ESIS\index{ESIS}, który wykorzystywany jest jako dane wejściowe przez
wiele narzędzi do przetwarzania i~formatowania dokumentów SGML.
Dodatkowymi, bardzo przydatnymi elementami pakietu \texttt{SP} są:
program \texttt{sgmlnorm} do normalizacji, program \texttt{sx} służący
do konwersji dokumentu SGML na XML oraz biblioteki programistyczne,
przydatne przy tworzeniu specjalistycznych aplikacji służących do
przetwarzania dokumentów SGML.

W~przypadku dokumentów XML publicznie dostępnych, parserów jest
w~chwili obecnej kilkadziesiąt. Do popularniejszych zaliczyć można
Microsoft Java XML Parser firmy Microsoft, LT XML firmy Language
Technology Group, Exapt oraz XP (James Clark)

\section{Wykorzystanie języków skryptowych}

\section{Wykorzystanie szablonów XSL}

Stosując wersję XML typu DocBook można wykorzystać szablony stylów
przygotowane w~standardzie XSL (autor N.~Walsh). W~chwili obecnej
są dostępne narzędzia umożliwiające przetworzenie dokumentów XML do
postaci drukowanej (Adobe PDF) oraz hipertekstowej (HTML).

Podobnie jak w~przypadku szablonów DSSSL, szablony stylów XSL są
sparametryzowane i~udokumentowane i~dzięki temu łatwe w~adaptacji. Do
zamiany dokumentu XML na postać prezentacyjną można wykorzystać jeden
z~dostępnych publicznie procesorów XSLT
(por.~tabela~\ref{zest:proces:xslt}).

\begin{table}[!htb]
\begin{tabular}{|l|l|l|} \hline
Nazwa & Autor      & Adres URL \\ \hline
\texttt{sablotron} & Ginger Alliance & \url{http://www.gingerall.com} \\ \hline
\texttt{Xt}        & J.~Clark & \url{http://www.jclark.com} \\ \hline
\texttt{4XSLT}     & FourThought & \url{http://www.fourthought.com} \\ \hline
\texttt{Saxon}     & Michael Kay &  \url{http://users.iclway.co.uk/mhkay/saxon} \\ \hline
\texttt{Xalan}     & Apache XML Project & \url{http://xml.apache.org} \\ \hline
\end{tabular}
\caption{Publicznie dostępne procesory XLST\label{zest:proces:xslt}}
\source{Opracowanie własne}
\end{table}

XSL:FO jest skomplikowanym językiem o~dużych możliwościach,
zawierającym ponad 50 różnych ,,obiektów formatujących'', począwszy od
najprostszych, takich jak prostokątne bloki tekstu poprzez wyliczenia,
tabele i~odsyłacze. Obiekty te można formatować wykorzystując przeszło
200 różnych właściwości (\emph{properties\/}), takich jak: kroje,
odmiany i~wielkości pisma, odstępy, kolory itp.
W~tym dokumencie przedstawione jest absolutne miniumum informacji
na temat standardu XSL:FO.

Cały dokument XSL:FO zawarty jest wewnątrz elementu \texttt{fo:root}.
Element ten zawiera (w~podanej niżej kolejności):

\begin{itemize}
\item dokładnie jeden element \texttt{fo:layout-master-set} zawierający
  szablony określające wygląd poszczególnych stron oraz sekwencji
  stron (te ostatnie są opcjonalne, ale typowo są definiowane);
\item zero lub więcej elementów \texttt{fo:declarations};
\item jeden lub więcej elementów \texttt{fo:page-sequance}
 zawierających treść formatowanego dokumentu wraz z~opisem
 jego sformatowania i~podziału na strony.
\end{itemize}

% zakończenie
\summary
Możliwości, jakie stoją przed archiwum prac magisterskich opartych na
XML-u, są ograniczone jedynie czasem, jaki należy poświęcić na pełną
implementację systemu. Nie ma przeszkód technologicznych do stworzenia
co najmniej równie doskonałego repozytorium, jak ma to miejsce w
przypadku ETD. Jeżeli chcemy w pełni uczestniczyć w rozwoju nowej ery
informacji, musimy szczególną uwagę przykładać do odpowiedniej
klasyfikacji i archiwizacji danych. Sądzę, że język XML znacznie to
upraszcza.

% załączniki (opcjonalnie):
\appendix
\chapter{Tytuł załącznika jeden}

Treść załącznika jeden.

\chapter{Tytuł załącznika dwa}

Treść załącznika dwa.

% literatura (obowiązkowo):
\bibliographystyle{unsrt}
\bibliography{xml}

% spis tabel (jeżeli jest potrzebny):
\listoftables

% spis rysunków (jeżeli jest potrzebny):
\listoffigures

\oswiadczenie

\end{document}
