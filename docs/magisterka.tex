\documentclass[brudnopis]{xmgr}
% Jeśli nowe rozdziały mają się zaczynać na stronach
% nieparzystych:
%\documentclass[openright]{xmgr}

\usepackage{epigraph}
\usepackage{url}

\defaultfontfeatures{Scale=MatchLowercase}
\setmainfont[Numbers=OldStyle,Ligatures=TeX]{Minion Pro}
\setsansfont[Numbers=OldStyle,Ligatures=TeX]{Myriad Pro}
% for fontspec version < 2.0
%\setmainfont[Numbers=OldStyle,Mapping=tex-text]{Minion Pro}
\setsansfont[Numbers=OldStyle,Mapping=tex-text]{Myriad Pro}
%\setmonofont[Scale=0.75]{Monaco}

% Opcjonalnie identyfikator dokumentu
% drukowany tylko z włączoną opcją 'brudnopis':
\wersja   {wersja wstępna [\ymdtoday]}

\author   {Piotr Lewandowski}
\nralbumu {215575}
\email    {poczta@piotrl.net}

\title    {Analiza wpływu nawyków muzycznych na aktywności wykonywane przy komputerze}
\date     {2017}
\miejsce  {Gdańsk}

\opiekun  {dr W. Bzyl}

% dodatkowe polecenia
%\renewcommand{\filename}[1]{\texttt{#1}}
%\definecolor{stress}{cmyk}{0,1,0.13,0} % RubineRed
%\definecolor{topic}{cmyk}{0.98,0.13,0,0.43} % MidnightBlue

\begin{document}

% streszczenie
\begin{abstract}
    W ramach pracy magisterskiej napisano aplikację internetową,
    wdrożoną w chmurze Digital Ocean pod adresem http://tbd.digitalocean.com/
    z przygotowanymi danymi testowymi pod kontem (user: test, login: test).

    Aplikacja umożliwia agregację danych użytkownika z dwóch serwisów,
    RescueTime — lista aktywności oraz
    Last.fm — lista odsłuchiwanych utworów.
    Pobrane dane są połączone na wspólnej osi czasu i wizualizowane pod różnymi względami za pomocą wykresów oraz tabel.

    Mechanizm agregacji napisany jest w języku Java i frameworku Spring,
    dane przechowywane są w bazie danych PostgreSQL,
    a warstwa wizualna została stworzona w JavaScript
    z użyciem biblioteki generującej wykresy — C3.js
    oraz frameworka Material Design.

    Kod znajduje się w prywatnym repozytorium GIT (pod adresem \url{https://github.com/piotrl/master-thesis}),
    pytania o dostęp lub o pracę można kierować na mail: poczta@piotrl.net.
\end{abstract}

% słowa kluczowe
\keywords{
    Aggregation,
    Data Integration,
    Data Visualisation,
    PostgreSQL,
    JavaScript,
    Java
}

% tytuł i spis treści
\maketitle

% wstęp
\introduction

\epigraph{Without deviation from the norm, progress is not possible}{\textit{Frank Zappa}}

[Work in progress]

Podczas pracy często słucham muzyki, by się odciąć od zewnętrznych dźwięków oraz odpowiednio nastroić.

Spotyka się na artykuły twierdzące, że odpowiednio dobrana muzyka wpływa na lepszą koncentrację.
Popularnym przykładem jest wystąpienie Willa Henshalla na konferencji TEDx,
którą w momencie pisania pracy obejrzano ponad 400 tysięcy razy:

\begin{figure}
  \includegraphics[width=\linewidth]{fig/tedx-will-hensell.png}
  \caption{Tedx Bruksela 04.11.2013: }
  \label{fig:Tedx Bruksela 04.11.2013: }
\end{figure}

Jednym z pytań w corocznej sondzie serwisu StackOverflow jest pytanie o muzyczne preferencje podczas programowania.
Na to pytanie odpowiedziało XX tysięcy ludzi (TODO: Źródło).

\begin{figure}
  \includegraphics[width=\linewidth]{fig/stack_overflow_music.png}
  \caption{Pytanie o preferncje muzyczne na StackOverflow}
  \label{fig:Pytanie o preferncje muzyczne na StackOverflow}
\end{figure}


\chapter{Agregacja danych}

    \section{Wybór dostawców danych}

    Podstawą działania aplikacji są dane z komputera użytkownika,
    lista aktywności wykonywanych w danej jednostce czasu oraz odsłuchiwana muzyka.
    Aby zdobyć te dane, postanowiłem nie pisać programu śledzącego procesy (daemon),
    lecz skorzystać z gotowych rozwiązań.

        \section*{RescueTime}

        RescueTime to program, który nasłuchuje na aktualnie aktywne procesy
        i raz na minutę wysyła te dane do serwisu internetowego RescueTime.com.
        Podstawową cechą tego serwisu jest automatyczne kategoryzowanie aktywności oraz ustawienie produktywności w 5 punktowej skali:
        „Bardzo produktywny”, „Produktywny”, „Neutralny”, „Rozpraszający” i „Bardzo rozpraszający”.

        Dla przykładu odtwarzacz wideo VLC jest przypisany do kategorii rozrywka ze statusem „Bardzo rozpraszający”,
        natomiast przeglądanie plików programem Windows Explorer jest w kategorii „Pomocnicze” ze statusem „Neutralny”.

        Dzięki temu jesteśmy w stanie sprawdzić nasz codzienny współczynnik produktywności.

        \begin{figure}
          \includegraphics[width=\linewidth]{fig/rescuetime-daily-dashboard.png}
          \caption{Codzienny raport aktywności RescueTime}
          \label{fig:RescueTime}
        \end{figure}

        Program jest w stanie rozpoznać nie tylko tytuły aktualnie otwartych okien,
        ale również w przypadku przeglądarki – adresy odwiedzanych stron internetowych,
        jest to bardzo ważne ze względu na dominującą ilość czasu spędzonego dziennie w samej przeglądarce.

        \section*{last.fm}

        Last.fm to serwis społecznościowy dla fanów muzyki,
        pomagający odkrywać nowych artystów na podstawie historii użytkownika.

        Last.fm zbiera dane użytkowników za pomocą wtyczek, instalowanych do odtwarzaczy muzycznych, np. Spotify.
        Każdy odsłuchany w co najmniej w połowie utwór jest logowany do bazy danych,
        w której użytkownicy na podstawie meta-tagów przypisują go do artysty, płyty oraz gatunku muzycznego.

        \begin{figure}
          \includegraphics[width=\linewidth]{fig/lastfm-now-listening.png}
          \caption{Raport słuchanej muzyki w last.fm}
          \label{fig:Last.fm}
        \end{figure}

        \section*{Komunikacja z zewnętrznymi serwisami}

    \section{Mechanizm agregacji danych}

        Oba z wymienionych serwisów udostępniają część danych, które użytkownik zgromadził na swoim koncie.
        Dostęp ten jest realizowany przez API, po okazaniu odpowiednich tokenów uwierzytelniających, mamy dostęp do danych użytkownika.

        Przed analizą, konieczne jest pobranie danych do własnej bazy danych oraz stworzenie modelu relacji pomiędzy nimi.
        Proces pobierania nazywam agregacją danych i jest on podstawą działania aplikacji,
        ponieważ to w nim zdobywamy interesujące nas informacje.

        Aby aplikacja nie była jednorazowego użytku, należy dla każdego użytkownika dostarczać kolejne porcje danych.
        Rodzi to szereg problemów oraz wyznacza pewne cechy, które napisany mechanizm musi spełniać.

        \subsection*{Niezależność}
        Podczas pobierania danych dla jednego użytkownika może wystąpić błąd, np. poprzez tymczasową niedostępność API,
        nie powinien on wpłynąć na agregacje innych użytkowników, ani na działanie mechanizmu w przyszłości dla tego samego użytkownika
        (Nie powinien ciągle powtarzać błędnych zapytań).

        \subsection*{Ciągłość}
        Celem codziennego uruchamiania agregacji jest pobranie najnowszych danych do systemu,
        API pozwalają nam na wybranie tylko małego fragmentu danych, więc istotne jest by zacząć agregację od momentu ostatniej
        dla danego użytkownika.

        \subsection*{Powtarzalność}

    \section{Powszechne problemy przy komunikacji z API}

        \subsection*{Ograniczenia dostawcy API}

        \subsection*{Autoryzacja}

        \subsection*{Braki dostępu}

        \subsection*{Zepsute dane}
        API nie gwarantuje istnienia ani poprawności wszystkich informacji.
        Przykładowo last.fm do identyfikacji utworu dostarcza jego identyfikator i pełną nazwę.
        Zawsze bezpieczniej jest korzystać z identyfikatora, ponieważ w nazwie mogą występować niebezpieczne dane np. znaki UTF-8,
        jednak istnieją, rekordy dla których identyfikator jest pusty.

        \subsection*{Wydajność}
        Dobrze jest kopiować wszystkie powtarzalne informacje do własnej bazy danych,
        dzięki temu możemy ograniczyć ilość żądań do API, które może mieć limity.

\chapter{Analiza danych}

    \section{Opis zebranych danych}

    \section{Data fusion - Łączenie danych w czasie}

        \subsection*{Korekta}

        \subsection*{Integracja dwóch źródeł danych}

     \section{Budowa raportów na podstawie szeregów czasowych}

\chapter{Wizualizacja danych}

     \section{Jeden zbiór danych - wiele interpretacji}

     \section{Wybór odpowiednich metod wizualizacji danych}

     \section{Analiza danych z punktu widzenia użytkownika}

% zakończenie
\summary
[Work in progress]

% literatura (obowiązkowo):
\bibliographystyle{unsrt}
\bibliography{xml}

% spis tabel (jeżeli jest potrzebny):
\listoftables

% spis rysunków (jeżeli jest potrzebny):
\listoffigures

\oswiadczenie

\end{document}
